\section{Match-pair Generation}
\begin{itemize}
\item Define match pair problem
\item Review previous algorithm
\item Motivate extension to tags
\item How to handle communicators?
\item Intuitive description of algorithm design
\item Introduction to pseudocode
\end{itemize}

%!TEX root = paper.tex

% Inputs
\newcommand{\inSends}{Sends}
\newcommand{\inRecvs}{Receives}
% Constants
\newcommand{\anySrc}{\ast}
\newcommand{\anyTag}{ANY}
\newcommand{\emptyArray}{int[]}
% Functions
\newcommand{\filterSends}[2]{\Call{Matching}{#1,#2}}
\newcommand{\take}[2]{\Call{Take}{#1,#2}}
\newcommand{\getSrc}[1]{\Call{GetSource}{#1}}
\newcommand{\getTag}[1]{\Call{GetTag}{#1}}
\newcommand{\getTop}[1]{\Call{Top}{#1}}
\newcommand{\dequeue}[1]{\Call{Dequeue}{#1}}
% Variables
\newcommand{\srcWild}{send\_wild}
\newcommand{\tagWild}{tag\_wild}
\newcommand{\bothWild}{both\_wild}
\newcommand{\topRecv}{receive}
\newcommand{\sends}{matches}
\newcommand{\recvSource}{source}
\newcommand{\recvTag}{tag}
\newcommand{\matchList}{send\_list}
\newcommand{\matchSource}{match\_source}
\newcommand{\matchTag}{match\_tag}
\newcommand{\wildCount}{wildcards}
\newcommand{\matchQueues}{queues}
\newcommand{\matchPairs}{match\_pairs}

\begin{algorithm*}
\begin{algorithmic}[5]
\Function {GenerateMatchPairs}{$ \inRecvs,\inSends $}
	\State {$ \matchPairs \gets \emptyset $}
	\State {$ \srcWild \gets \emptyArray $}
	\State {$ \tagWild \gets \emptyArray $}
	\State {$ \bothWild \gets 0 $}
	\ForAll {$ \topRecv \in \inRecvs $}
		\State {$ \sends \gets \emptyset $}
		\State {$ \recvSource \gets \getSrc{\topRecv} $}
		\State {$ \recvTag \gets \getTag{\topRecv} $}
		\State {$ \matchQueues \gets \filterSends{\inSends}{\topRecv} $}
		\If {$(\recvSource = \anySrc) \land (\recvTag = \anyTag) $}
			\State {$ \bothWild \gets \bothWild + 1 $}
		\ElsIf {$ (\recvSource = \anySrc) $}
			\State {$ \srcWild[\recvTag] \gets \srcWild[\recvTag] + 1 $}
		\ElsIf {$ (\recvTag = \anyTag) $}
			\State {$ \tagWild[\recvSource] \gets \tagWild[\recvSource] + 1 $}
		\Else
			\State {$ \matchList \gets \getTop{\matchQueues} $}
			\State {$ \sends \gets \dequeue{\matchList} $}
		\EndIf
		\ForAll {$ \matchList \in \matchQueues $}
			\State {$ \matchSource \gets \getSrc{\matchList} $}
			\State {$ \matchTag \gets \getTag{\matchList} $}
			\State {$ \wildCount \gets (\srcWild[\matchTag] + \tagWild[\matchSource] + \bothWild) $}
			\State {$ \sends \gets \sends \cup \take{\matchList}{\wildCount} $}
		\EndFor
		\State {$ \matchPairs \gets \matchPairs[\topRecv \mapsto \sends] $}
	\EndFor
	\Return {$ \matchPairs $}
\EndFunction
\end{algorithmic}
\end{algorithm*}

%%% Local Variables:
%%% mode: latex
%%% TeX-master: "paper"
%%% End:


\subsection{Algorithm analysis}
Let the program have $m$ processes, with at most $n$ sends or receives in a
process.

The best case for this algorithm is a program with only deterministic receives.
In this case each receive will have one and only one matching receive. This
means that analyzing a single process is $\mathcal{O}(n)$, so analysis of the
entire program is $\mathcal{O}(mn)$.

Conversely, the worse case is that all sends have a unique tag within their
process and all receives are both source non-deterministic and tag
non-deterministic. In this case there will be $\mathcal{O}(mn)$ matches per
receive, and therefore $\mathcal{O}(m^2n^2)$ match-pairs overall. Many of these
can be trivially discarded by remembering that two sends from the same process
that might match the same receive must be received in program order. This would
reduce the problem to $\mathcal{O}(m)$ matches per receive, or
$\mathcal{O}(m^2n)$ overall.

%%% Local Variables:
%%% mode: latex
%%% TeX-master: "paper"
%%% End:
