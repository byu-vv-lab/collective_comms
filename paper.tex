%!TEX program = xelatex

\documentclass[preprint]{sigplanconf}

% The following \documentclass options may be useful:

% preprint      Remove this option only once the paper is in final form.
% 10pt          To set in 10-point type instead of 9-point.
% 11pt          To set in 11-point type instead of 9-point.
% numbers       To obtain numeric citation style instead of author/year.

\usepackage{amsmath}
\usepackage[noend]{algpseudocode}
\usepackage{algorithm}

\newcommand{\cL}{{\cal L}}

\begin{document}

\special{papersize=8.5in,11in}
\setlength{\pdfpageheight}{\paperheight}
\setlength{\pdfpagewidth}{\paperwidth}

% \conferenceinfo{CONF 'yy}{Month d--d, 20yy, City, ST, Country}
% \copyrightyear{20yy}
% \copyrightdata{978-1-nnnn-nnnn-n/yy/mm}
% \copyrightdoi{nnnnnnn.nnnnnnn}

% Uncomment the publication rights you want to use.
%\publicationrights{transferred}
%\publicationrights{licensed}     % this is the default
%\publicationrights{author-pays}

% \titlebanner{banner above paper title}        % These are ignored unless
% \preprintfooter{short description of paper}   % 'preprint' option specified.

\title{Verification of MPI programs with tags}

\authorinfo{Joshua T. Asplund}
           {Brigham Young University}
           {joshuata@cs.byu.edu}
% \authorinfo{Name2\and Name3}
%            {Affiliation2/3}
%            {Email2/3}

\maketitle

\begin{abstract}

The Message Passing Interface (MPI) is a common programming model in high
performance computing (HPC). The parallel nature and large scale of MPI
programs makes them a valuable target for program verification. Existing tools
approach the problem by verifying a small subset of the MPI specification.
Specifically, previous static analyses only addressed point-to-point
communication without tags. This paper presents an extension to an existing
SMT-based approach that adds support for tagged messages and multiple
communicators. We then show how to leverage this extension to verify programs
that use MPI collective communication. The paper also presents an efficient
algorithm that approximates the input match pairs using simple counting for two
types of message non-determinism. The experimental results demonstrate that the
new SMT encoding is capable of precisely reasoning about the behavior in a set
of benchmarks.

\end{abstract}

%%% Local Variables:
%%% mode: latex
%%% TeX-master: "paper"
%%% End:


% \category{CR-number}{subcategory}{third-level}

% general terms are not compulsory anymore,
% you may leave them out

% \keywords
% keyword1, keyword2

\section{Introduction}
This is the introduction
Here is more introduction

% TODO: How to generate CTP

%%% Local Variables:
%%% mode: latex
%%% TeX-master: "paper"
%%% End:

\section{Pseudocode}
This is the pseudocode

%!TEX root = paper.tex

% Inputs
\newcommand{\inSends}{Sends}
\newcommand{\inRecvs}{Receives}
% Constants
\newcommand{\anySrc}{\ast}
\newcommand{\anyTag}{ANY}
\newcommand{\emptyArray}{int[]}
% Functions
\newcommand{\filterSends}[2]{\Call{Matching}{#1,#2}}
\newcommand{\take}[2]{\Call{Take}{#1,#2}}
\newcommand{\getSrc}[1]{\Call{GetSource}{#1}}
\newcommand{\getTag}[1]{\Call{GetTag}{#1}}
\newcommand{\getTop}[1]{\Call{Top}{#1}}
\newcommand{\dequeue}[1]{\Call{Dequeue}{#1}}
% Variables
\newcommand{\srcWild}{send\_wild}
\newcommand{\tagWild}{tag\_wild}
\newcommand{\bothWild}{both\_wild}
\newcommand{\topRecv}{receive}
\newcommand{\sends}{matches}
\newcommand{\recvSource}{source}
\newcommand{\recvTag}{tag}
\newcommand{\matchList}{send\_list}
\newcommand{\matchSource}{match\_source}
\newcommand{\matchTag}{match\_tag}
\newcommand{\wildCount}{wildcards}
\newcommand{\matchQueues}{queues}
\newcommand{\matchPairs}{match\_pairs}

\begin{algorithm*}
\begin{algorithmic}[5]
\Function {GenerateMatchPairs}{$ \inRecvs,\inSends $}
	\State {$ \matchPairs \gets \emptyset $}
	\State {$ \srcWild \gets \emptyArray $}
	\State {$ \tagWild \gets \emptyArray $}
	\State {$ \bothWild \gets 0 $}
	\ForAll {$ \topRecv \in \inRecvs $}
		\State {$ \sends \gets \emptyset $}
		\State {$ \recvSource \gets \getSrc{\topRecv} $}
		\State {$ \recvTag \gets \getTag{\topRecv} $}
		\State {$ \matchQueues \gets \filterSends{\inSends}{\topRecv} $}
		\If {$(\recvSource = \anySrc) \land (\recvTag = \anyTag) $}
			\State {$ \bothWild \gets \bothWild + 1 $}
		\ElsIf {$ (\recvSource = \anySrc) $}
			\State {$ \srcWild[\recvTag] \gets \srcWild[\recvTag] + 1 $}
		\ElsIf {$ (\recvTag = \anyTag) $}
			\State {$ \tagWild[\recvSource] \gets \tagWild[\recvSource] + 1 $}
		\Else
			\State {$ \matchList \gets \getTop{\matchQueues} $}
			\State {$ \sends \gets \dequeue{\matchList} $}
		\EndIf
		\ForAll {$ \matchList \in \matchQueues $}
			\State {$ \matchSource \gets \getSrc{\matchList} $}
			\State {$ \matchTag \gets \getTag{\matchList} $}
			\State {$ \wildCount \gets (\srcWild[\matchTag] + \tagWild[\matchSource] + \bothWild) $}
			\State {$ \sends \gets \sends \cup \take{\matchList}{\wildCount} $}
		\EndFor
		\State {$ \matchPairs \gets \matchPairs[\topRecv \mapsto \sends] $}
	\EndFor
	\Return {$ \matchPairs $}
\EndFunction
\end{algorithmic}
\end{algorithm*}

%%% Local Variables:
%%% mode: latex
%%% TeX-master: "paper"
%%% End:


%%% Local Variables:
%%% mode: latex
%%% TeX-master: "paper"
%%% End:

\section{Collective Communication Operations}

According to the MPI specification, collective communications can be expressed
using point-to-point operations and a reserved tag space. Most of the
derivations follow similar patterns, so we will present a few specific examples
rather than enumerating all of them.

\subsection{Example: broadcast}
The MPI broadcast operation is a one-to-many communication. It takes the message
and copies it into the receive buffer of each process in the group. This is
simply encoded by executing a send from the source to each process that is a
member of the group on the group's communicator. This is paired with a blocking
receive on each destination process in the group. We use non-blocking sends
grouped together with all of the witnessing waits following the sends. This is
necessary since the runtime can order the sends as it wishes in order to avoid
locking.

\subsection{Example: gather}
The gather operation is the complement of broadcast. The send buffer of each
process in the group is copied into the receive buffer of the destination
process. Each send is blocking, and the receives are non-blocking, with all of
the waits immediately following the last receive.

\subsection{Example: barrier}
A barrier guarantees that all processes in the group are synchronized at that
point in the schedule. In our encoding this is handled as a separate operation
as other derivations would require adding many more communication operations.

%%% Local Variables:
%%% mode: latex
%%% TeX-master: "paper"
%%% End:

\section{Encoding}
This is the section on encoding

%%% Local Variables:
%%% mode: latex
%%% TeX-master: "paper"
%%% End:

\section{Empirical Study}
This is for our results


%%% Local Variables:
%%% mode: latex
%%% TeX-master: "paper"
%%% End:

\section{Related Work}
Related work


%%% Local Variables:
%%% mode: latex
%%% TeX-master: "paper"
%%% End:

\section{Conclusion and Future Research}
Conclusion and future research


%%% Local Variables:
%%% mode: latex
%%% TeX-master: "paper"
%%% End:


% \appendix
% \section{Appendix Title}

% This is the text of the appendix, if you need one.

% \acks

% Acknowledgments, if needed.

% We recommend abbrvnat bibliography style.

% \bibliographystyle{abbrvnat}

% The bibliography should be embedded for final submission.

% \begin{thebibliography}{}
% \softraggedright

% \end{thebibliography}


\end{document}
%%% Local Variables:
%%% mode: latex
%%% TeX-master: t
%%% End:
