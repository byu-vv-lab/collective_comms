\section{Send Modes}

\subsection{Introduction}
% \subsubsection{What are send modes?}
The MPI specification defines 4 distinct send modes: standard, buffered,
synchronous, and ready. These define when the runtime marks a send as complete.
Additionally, each of these modes has both a blocking and a non-blocking
variant.
% \subsubsection{What bugs can they cause?}
Because each one behaves differently, they can introduce subtle bugs that
are difficult to reproduce. 
% \subsubsection{Why are they hard to verify?}
This also introduces 8 different variations of the send command, each of which
needs its own unique encoding strategy.

% \subsubsection{Both blocking and non-blocking variants of each mode}
To remove redundant operations from our encoding, we consider a blocking receive
as a non-blocking receive followed immediately on the same process by its
witnessing receive.

\subsection{Standard and Buffered Send}
% \subsubsection{Semantics}
Both the standard and buffered send are considered complete as soon as the
runtime has acknowledged the message, with no guarantees about message
reception. We also make the simplifying assumption that buffer space is
unlimited. This allows us to use the same encoding for both modes.
% \subsubsection{Equivalence of encodings}

\subsection{Synchronous Send}
% \subsubsection{Semantics}
The synchronous send mode requires that the message is received before the send
is considered complete. This forces schedules to synchronize, allowing more
control over how processes interleave.
% \subsubsection{Extension of send encoding}

We extend our encoding to handle these semantics by requiring that the matching
receive happens before the send's witnessing receive.
% TODO: Example

\subsection{Ready Send}
% \subsubsection{Semantics}
The ready send presents the greatest to the encoding. It is a performance
optimization to allow the runtime to copy the message contents directly into the
receiver's message buffer. To achieve this, it requires that the matching receive
be posted prior to the send. If a process attempts to send a message that does
not have a matching send waiting for it, the send results in an error.

% \subsubsection{Extension of send encoding}
To check for ready send compatibility, we add a clause where the send happens
before its matching receive. If we can find a valid schedule that satisfies this
clause then the program is not ready send compatible.
% TODO: Example

%%% Local Variables:
%%% mode: latex
%%% TeX-master: "paper"
%%% End:
